\subsection{Lecture 1} September 14, 2015

The website of the course:
	\url{http://www.math.toronto.edu/~jkamnitz/courses/mat347/index.html}

Lecture time:
Monday, Wednesday 11-12 at BA1200
Friday 10-12 at UC52

Topics: group (symmetries).

\subsubsection{Definitions}
A binary operation on a set $G$ is a map $\bullet$,
	$G \times G \to G$ (aka. $(a, b) \to a \cdot b$).

It's called \kw{associative} if $a \cdot ( b \cdot c ) = (a \cdot b) \cdot c$.

It's called \kw{commutative} if $a \cdot b = b \cdot a$.

It has an \kw{identity} if $\exists e \in G$ st. $a \cdot e = a = e \cdot a$.

It has \kw{inverses} if for all $a \in G$ there exists $a^{-1} \in G$ such that
$a \cdot a^{-1} = e = a^{-1} \cdot a$.

A \kw{group} is a set with a binary operation which is associative and has an identity and inverses.

A commutative group is called an \kw{abelian group}.

A \kw{permutation} of a set $X$ is a bijection $\sigma : X \to X$.

\subsubsection{Examples}

\paragraph{Ex1}
$\bZ$ with addition, $\bR$ with addition, and
$\bR^\times=\bR\setminus\{0\}$ with multiplication are all examples of abelian group.

\paragraph{Ex2}
``The main example''

For any set $X$ define $S_X = \{\textrm{permutation of the set }X\}$.
The binary operation on $S_X$ \kw{composition}.

If $X=\{1, \ldots, n\}$, then define $S_n$ to be $S_{\{1, \ldots, n\}}$.

\paragraph{Ex3}
Let $V$ be a vector space over any fields $(\bR, \bQ, \bC, \bF_2, \bF_p)$.

$(V, +)$ is a group, $S_V$ is also a group.

An more interesting example is
\[GL(V) := \{\textrm{all invertable linear operator } T:V\to V\}\]
note that $GL(V) \subseteq S_V$

Another interesting example is
\[GL_n(k) := \{\textrm{invertable nxn matrices with entries in }k\}\]

\paragraph{Ex4}
If V catties a symmetric bilinear form $\langle,\rangle$,
then we define $O(V, \langle,\rangle) = \{\textrm{invertable }T:V\to V\textrm{ st }\langle Tv,Tw\rangle = \langle v,w\rangle\textrm{ for all } v,w\in V\}$.

For example, $O(\bR^n, \langle,\rangle) = O_n(\bR) = \{A^\mathrm{invertable} \textrm{ nxn matrices st } AA^{-1} = I$. When $n=2$, $O_2(\bR)$ is rotation and reflection of $\bR^2$.

\begin{prop}
If $G$ is a group and $a, b, c, d \in G$ and $ab=ac$ then $b=c$.
\end{prop}
\begin{proof}
Start with $ab=ac$ multiply both side by $a^-1$,
then
\begin{align*}
ab &= ac \\
a^{-1}(ab) &= a^{-1}(ac) \\
(a^{-1}a)b &= (a^{-1}a)c \\
b &= c \\
\end{align*}
\end{proof}
\begin{remark}
Special case: every group has unique identity.
\end{remark}



